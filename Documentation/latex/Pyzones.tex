%% Generated by Sphinx.
\def\sphinxdocclass{report}
\documentclass[letterpaper,10pt,english]{sphinxmanual}
\ifdefined\pdfpxdimen
   \let\sphinxpxdimen\pdfpxdimen\else\newdimen\sphinxpxdimen
\fi \sphinxpxdimen=.75bp\relax

\PassOptionsToPackage{warn}{textcomp}
\usepackage[utf8]{inputenc}
\ifdefined\DeclareUnicodeCharacter
% support both utf8 and utf8x syntaxes
\edef\sphinxdqmaybe{\ifdefined\DeclareUnicodeCharacterAsOptional\string"\fi}
  \DeclareUnicodeCharacter{\sphinxdqmaybe00A0}{\nobreakspace}
  \DeclareUnicodeCharacter{\sphinxdqmaybe2500}{\sphinxunichar{2500}}
  \DeclareUnicodeCharacter{\sphinxdqmaybe2502}{\sphinxunichar{2502}}
  \DeclareUnicodeCharacter{\sphinxdqmaybe2514}{\sphinxunichar{2514}}
  \DeclareUnicodeCharacter{\sphinxdqmaybe251C}{\sphinxunichar{251C}}
  \DeclareUnicodeCharacter{\sphinxdqmaybe2572}{\textbackslash}
\fi
\usepackage{cmap}
\usepackage[T1]{fontenc}
\usepackage{amsmath,amssymb,amstext}
\usepackage{babel}
\usepackage{times}
\usepackage[Bjarne]{fncychap}
\usepackage{sphinx}

\fvset{fontsize=\small}
\usepackage{geometry}

% Include hyperref last.
\usepackage{hyperref}
% Fix anchor placement for figures with captions.
\usepackage{hypcap}% it must be loaded after hyperref.
% Set up styles of URL: it should be placed after hyperref.
\urlstyle{same}
\addto\captionsenglish{\renewcommand{\contentsname}{Contents:}}

\addto\captionsenglish{\renewcommand{\figurename}{Fig.\@ }}
\makeatletter
\def\fnum@figure{\figurename\thefigure{}}
\makeatother
\addto\captionsenglish{\renewcommand{\tablename}{Table }}
\makeatletter
\def\fnum@table{\tablename\thetable{}}
\makeatother
\addto\captionsenglish{\renewcommand{\literalblockname}{Listing}}

\addto\captionsenglish{\renewcommand{\literalblockcontinuedname}{continued from previous page}}
\addto\captionsenglish{\renewcommand{\literalblockcontinuesname}{continues on next page}}
\addto\captionsenglish{\renewcommand{\sphinxnonalphabeticalgroupname}{Non-alphabetical}}
\addto\captionsenglish{\renewcommand{\sphinxsymbolsname}{Symbols}}
\addto\captionsenglish{\renewcommand{\sphinxnumbersname}{Numbers}}

\addto\extrasenglish{\def\pageautorefname{page}}

\setcounter{tocdepth}{2}



\title{Pyzones Documentation}
\date{Feb 11, 2019}
\release{11/02/2019}
\author{Craig Henry}
\newcommand{\sphinxlogo}{\vbox{}}
\renewcommand{\releasename}{Release}
\makeindex
\begin{document}

\pagestyle{empty}
\sphinxmaketitle
\pagestyle{plain}
\sphinxtableofcontents
\pagestyle{normal}
\phantomsection\label{\detokenize{index::doc}}



\chapter{PyZones}
\label{\detokenize{source/modules:pyzones}}\label{\detokenize{source/modules::doc}}

\section{pyzones module}
\label{\detokenize{source/pyzones:module-pyzones}}\label{\detokenize{source/pyzones:pyzones-module}}\label{\detokenize{source/pyzones::doc}}\index{pyzones (module)@\spxentry{pyzones}\spxextra{module}}\index{Circle (class in pyzones)@\spxentry{Circle}\spxextra{class in pyzones}}

\begin{fulllineitems}
\phantomsection\label{\detokenize{source/pyzones:pyzones.Circle}}\pysiglinewithargsret{\sphinxbfcode{\sphinxupquote{class }}\sphinxcode{\sphinxupquote{pyzones.}}\sphinxbfcode{\sphinxupquote{Circle}}}{\emph{centre}, \emph{radius}}{}
Bases: {\hyperref[\detokenize{source/pyzones:pyzones.Shape}]{\sphinxcrossref{\sphinxcode{\sphinxupquote{pyzones.Shape}}}}}

A geometric class for a circle
\begin{description}
\item[{centre}] \leavevmode{[}list{]}
A list of coordinates (x, y) describing the centre of the circle

\item[{radius}] \leavevmode{[}float{]}
The radius of the circle

\end{description}
\index{centre (pyzones.Circle attribute)@\spxentry{centre}\spxextra{pyzones.Circle attribute}}

\begin{fulllineitems}
\phantomsection\label{\detokenize{source/pyzones:pyzones.Circle.centre}}\pysigline{\sphinxbfcode{\sphinxupquote{centre}}}
A list of coordinates (x, y) describing the centre of the circle
\begin{quote}\begin{description}
\item[{Returns}] \leavevmode
(x, y) of the centre of the circle

\item[{Return type}] \leavevmode
list

\end{description}\end{quote}

\end{fulllineitems}

\index{get\_circular\_points() (pyzones.Circle method)@\spxentry{get\_circular\_points()}\spxextra{pyzones.Circle method}}

\begin{fulllineitems}
\phantomsection\label{\detokenize{source/pyzones:pyzones.Circle.get_circular_points}}\pysiglinewithargsret{\sphinxbfcode{\sphinxupquote{get\_circular\_points}}}{\emph{start\_angle}, \emph{end\_angle}, \emph{num\_points}, \emph{radius}, \emph{decimal\_places=None}}{}
Create a list of points between the user defined start and end angles on the perimeter of a new circle sharing
the centre point of this circle with a different radius
\begin{quote}\begin{description}
\item[{Parameters}] \leavevmode\begin{itemize}
\item {} 
\sphinxstyleliteralstrong{\sphinxupquote{start\_angle}} (\sphinxstyleliteralemphasis{\sphinxupquote{float}}) \textendash{} Position at which the list of points should begin

\item {} 
\sphinxstyleliteralstrong{\sphinxupquote{end\_angle}} (\sphinxstyleliteralemphasis{\sphinxupquote{float}}) \textendash{} Position at which the list of points should end

\item {} 
\sphinxstyleliteralstrong{\sphinxupquote{num\_points}} (\sphinxstyleliteralemphasis{\sphinxupquote{int}}) \textendash{} Number of points

\item {} 
\sphinxstyleliteralstrong{\sphinxupquote{radius}} (\sphinxstyleliteralemphasis{\sphinxupquote{float}}) \textendash{} Radius of the circle on which the points are placed

\item {} 
\sphinxstyleliteralstrong{\sphinxupquote{decimal\_places}} (\sphinxstyleliteralemphasis{\sphinxupquote{int}}) \textendash{} Number of decimal places the coordinates are returned with - None: there is no rounding

\end{itemize}

\item[{Returns}] \leavevmode
An array of points (x,y) evenly spaced on the perimeter of the new circle between the start and end angles

\item[{Return type}] \leavevmode
numpy.ndarray

\end{description}\end{quote}

\end{fulllineitems}

\index{get\_grid() (pyzones.Circle method)@\spxentry{get\_grid()}\spxextra{pyzones.Circle method}}

\begin{fulllineitems}
\phantomsection\label{\detokenize{source/pyzones:pyzones.Circle.get_grid}}\pysiglinewithargsret{\sphinxbfcode{\sphinxupquote{get\_grid}}}{\emph{spacing}, \emph{alpha=2}}{}
Create a grid of points spaced uniformly across the circle using the sunflower seed arrangement algorithm
\begin{quote}\begin{description}
\item[{Parameters}] \leavevmode\begin{itemize}
\item {} 
\sphinxstyleliteralstrong{\sphinxupquote{spacing}} (\sphinxstyleliteralemphasis{\sphinxupquote{float}}) \textendash{} Approximate spacing between points in the grid

\item {} 
\sphinxstyleliteralstrong{\sphinxupquote{alpha}} (\sphinxstyleliteralemphasis{\sphinxupquote{float}}) \textendash{} Determines the evenness of the boundary - 0 is jagged, 2 is smooth. Above 2 is not recommended

\end{itemize}

\item[{Returns}] \leavevmode
A list of points (x,y) uniformly spaced across the circle

\item[{Return type}] \leavevmode
numpy.ndarray

\end{description}\end{quote}

\end{fulllineitems}

\index{get\_perimeter() (pyzones.Circle method)@\spxentry{get\_perimeter()}\spxextra{pyzones.Circle method}}

\begin{fulllineitems}
\phantomsection\label{\detokenize{source/pyzones:pyzones.Circle.get_perimeter}}\pysiglinewithargsret{\sphinxbfcode{\sphinxupquote{get\_perimeter}}}{\emph{start\_angle}, \emph{end\_angle}, \emph{num\_points}, \emph{decimal\_places=None}}{}
Create a list of points between the user defined start and end angles on the perimeter of the circle
\begin{quote}\begin{description}
\item[{Parameters}] \leavevmode\begin{itemize}
\item {} 
\sphinxstyleliteralstrong{\sphinxupquote{start\_angle}} (\sphinxstyleliteralemphasis{\sphinxupquote{float}}) \textendash{} Position at which the list of points should begin

\item {} 
\sphinxstyleliteralstrong{\sphinxupquote{end\_angle}} (\sphinxstyleliteralemphasis{\sphinxupquote{float}}) \textendash{} Position at which the list of points should end

\item {} 
\sphinxstyleliteralstrong{\sphinxupquote{num\_points}} (\sphinxstyleliteralemphasis{\sphinxupquote{int}}) \textendash{} Number of points

\item {} 
\sphinxstyleliteralstrong{\sphinxupquote{decimal\_places}} (\sphinxstyleliteralemphasis{\sphinxupquote{int}}) \textendash{} Number of decimal places the coordinates are returned with - None: there is no rounding

\end{itemize}

\item[{Returns}] \leavevmode
A list of points (x,y) evenly spaced on the perimeter of the shape between the start and end angles

\item[{Return type}] \leavevmode
numpy.ndarray

\end{description}\end{quote}

\end{fulllineitems}

\index{is\_point\_inside() (pyzones.Circle method)@\spxentry{is\_point\_inside()}\spxextra{pyzones.Circle method}}

\begin{fulllineitems}
\phantomsection\label{\detokenize{source/pyzones:pyzones.Circle.is_point_inside}}\pysiglinewithargsret{\sphinxbfcode{\sphinxupquote{is\_point\_inside}}}{\emph{point}}{}
Check whether or not a point is inside the circle
\begin{quote}\begin{description}
\item[{Parameters}] \leavevmode
\sphinxstyleliteralstrong{\sphinxupquote{point}} (\sphinxstyleliteralemphasis{\sphinxupquote{list}}) \textendash{} List/tuple of the coordinates (x, y) of a point

\item[{Returns}] \leavevmode
A bool stating whether or not the point is within the circle

\item[{Return type}] \leavevmode
bool

\end{description}\end{quote}

\end{fulllineitems}

\index{radius (pyzones.Circle attribute)@\spxentry{radius}\spxextra{pyzones.Circle attribute}}

\begin{fulllineitems}
\phantomsection\label{\detokenize{source/pyzones:pyzones.Circle.radius}}\pysigline{\sphinxbfcode{\sphinxupquote{radius}}}
The radius of the circle
\begin{quote}\begin{description}
\item[{Returns}] \leavevmode
The radius

\item[{Return type}] \leavevmode
float

\end{description}\end{quote}

\end{fulllineitems}


\end{fulllineitems}

\index{Loudspeaker (class in pyzones)@\spxentry{Loudspeaker}\spxextra{class in pyzones}}

\begin{fulllineitems}
\phantomsection\label{\detokenize{source/pyzones:pyzones.Loudspeaker}}\pysiglinewithargsret{\sphinxbfcode{\sphinxupquote{class }}\sphinxcode{\sphinxupquote{pyzones.}}\sphinxbfcode{\sphinxupquote{Loudspeaker}}}{\emph{position=None}, \emph{colour=None}, \emph{look\_at=None}}{}
Bases: {\hyperref[\detokenize{source/pyzones:pyzones.SoundObject}]{\sphinxcrossref{\sphinxcode{\sphinxupquote{pyzones.SoundObject}}}}}

A loudspeaker to be used as a source in simulations of sound zones. Inherits from the sound object class.
\begin{description}
\item[{\_width}] \leavevmode{[}static float{]}
The width of rectangles used to represent loudspeakers when rendered in the soundfield

\item[{\_height}] \leavevmode{[}static float{]}
The height of rectangles used to represent loudspeakers when rendered in the soundfield

\end{description}
\begin{description}
\item[{colour}] \leavevmode{[}list{]}
A list of float values (r, g, b)

\item[{position}] \leavevmode{[}list{]}
A list of coordinates (x, y) describing the position of the loudspeaker

\item[{look\_at}] \leavevmode{[}list{]}
A list of coordinates (x, y) describing the position the loudspeaker faces

\item[{q}] \leavevmode{[}list{]}
The filter weight at each frequency most recently calculated and set

\end{description}
\index{height (pyzones.Loudspeaker attribute)@\spxentry{height}\spxextra{pyzones.Loudspeaker attribute}}

\begin{fulllineitems}
\phantomsection\label{\detokenize{source/pyzones:pyzones.Loudspeaker.height}}\pysigline{\sphinxbfcode{\sphinxupquote{height}}}
The height of rectangles used to represent loudspeakers when rendered in the soundfield
\begin{quote}\begin{description}
\item[{Returns}] \leavevmode
The height

\item[{Return type}] \leavevmode
float

\end{description}\end{quote}

\end{fulllineitems}

\index{look\_at (pyzones.Loudspeaker attribute)@\spxentry{look\_at}\spxextra{pyzones.Loudspeaker attribute}}

\begin{fulllineitems}
\phantomsection\label{\detokenize{source/pyzones:pyzones.Loudspeaker.look_at}}\pysigline{\sphinxbfcode{\sphinxupquote{look\_at}}}
A list of coordinates (x, y) describing the position the loudspeaker faces
\begin{quote}\begin{description}
\item[{Returns}] \leavevmode
a list of coordinates (x, y)

\item[{Return type}] \leavevmode
list

\end{description}\end{quote}

\end{fulllineitems}

\index{width (pyzones.Loudspeaker attribute)@\spxentry{width}\spxextra{pyzones.Loudspeaker attribute}}

\begin{fulllineitems}
\phantomsection\label{\detokenize{source/pyzones:pyzones.Loudspeaker.width}}\pysigline{\sphinxbfcode{\sphinxupquote{width}}}
The width of rectangles used to represent loudspeakers when rendered in the soundfield
\begin{quote}\begin{description}
\item[{Returns}] \leavevmode
The width

\item[{Return type}] \leavevmode
float

\end{description}\end{quote}

\end{fulllineitems}


\end{fulllineitems}

\index{LoudspeakerArray (class in pyzones)@\spxentry{LoudspeakerArray}\spxextra{class in pyzones}}

\begin{fulllineitems}
\phantomsection\label{\detokenize{source/pyzones:pyzones.LoudspeakerArray}}\pysiglinewithargsret{\sphinxbfcode{\sphinxupquote{class }}\sphinxcode{\sphinxupquote{pyzones.}}\sphinxbfcode{\sphinxupquote{LoudspeakerArray}}}{\emph{*args}}{}
Bases: {\hyperref[\detokenize{source/pyzones:pyzones.SoundObjectArray}]{\sphinxcrossref{\sphinxcode{\sphinxupquote{pyzones.SoundObjectArray}}}}}

A container class for loudspeakers
\index{get\_q() (pyzones.LoudspeakerArray method)@\spxentry{get\_q()}\spxextra{pyzones.LoudspeakerArray method}}

\begin{fulllineitems}
\phantomsection\label{\detokenize{source/pyzones:pyzones.LoudspeakerArray.get_q}}\pysiglinewithargsret{\sphinxbfcode{\sphinxupquote{get\_q}}}{\emph{frequency\_index}}{}
Get the filter weight values for the loudspeaker array at the frequency index
\begin{quote}\begin{description}
\item[{Parameters}] \leavevmode
\sphinxstyleliteralstrong{\sphinxupquote{frequency\_index}} (\sphinxstyleliteralemphasis{\sphinxupquote{int}}) \textendash{} The index at which the relevant frequency is stored in the simulation’s frequency vector

\item[{Returns}] \leavevmode
The list of filter weights calculated for the relevant frequency

\item[{Return type}] \leavevmode
list

\end{description}\end{quote}

\end{fulllineitems}

\index{initialise\_q() (pyzones.LoudspeakerArray method)@\spxentry{initialise\_q()}\spxextra{pyzones.LoudspeakerArray method}}

\begin{fulllineitems}
\phantomsection\label{\detokenize{source/pyzones:pyzones.LoudspeakerArray.initialise_q}}\pysiglinewithargsret{\sphinxbfcode{\sphinxupquote{initialise\_q}}}{\emph{num\_frequencies}}{}
Initialise the filter weights of the loudspeakers in the loudspeaker array to one.
\begin{quote}\begin{description}
\item[{Parameters}] \leavevmode
\sphinxstyleliteralstrong{\sphinxupquote{num\_frequencies}} (\sphinxstyleliteralemphasis{\sphinxupquote{int}}) \textendash{} The number of frequencies in the simulation.

\end{description}\end{quote}

\end{fulllineitems}

\index{set\_q() (pyzones.LoudspeakerArray method)@\spxentry{set\_q()}\spxextra{pyzones.LoudspeakerArray method}}

\begin{fulllineitems}
\phantomsection\label{\detokenize{source/pyzones:pyzones.LoudspeakerArray.set_q}}\pysiglinewithargsret{\sphinxbfcode{\sphinxupquote{set\_q}}}{\emph{new\_q}, \emph{frequency\_index}}{}
Set the filter weight values for the loudspeaker array at the frequency index
\begin{quote}\begin{description}
\item[{Parameters}] \leavevmode\begin{itemize}
\item {} 
\sphinxstyleliteralstrong{\sphinxupquote{new\_q}} (\sphinxstyleliteralemphasis{\sphinxupquote{list}}) \textendash{} The list of filter weights

\item {} 
\sphinxstyleliteralstrong{\sphinxupquote{frequency\_index}} (\sphinxstyleliteralemphasis{\sphinxupquote{int}}) \textendash{} The index at which the relevant frequency is stored in the simulation’s frequency vector

\end{itemize}

\end{description}\end{quote}

\end{fulllineitems}


\end{fulllineitems}

\index{Metrics (class in pyzones)@\spxentry{Metrics}\spxextra{class in pyzones}}

\begin{fulllineitems}
\phantomsection\label{\detokenize{source/pyzones:pyzones.Metrics}}\pysiglinewithargsret{\sphinxbfcode{\sphinxupquote{class }}\sphinxcode{\sphinxupquote{pyzones.}}\sphinxbfcode{\sphinxupquote{Metrics}}}{\emph{frequencies}, \emph{metrics}}{}
Bases: \sphinxcode{\sphinxupquote{object}}

Contains metrics for acoustic contrast, effort, planarity and reproduction error. This class can be used to print
these metrics as well as being able to save them to a CSV and create a plot over frequency.
\begin{description}
\item[{\_metrics}] \leavevmode{[}numpy.ndarray{]}
An array of the metrics across frequency

\item[{frequencies}] \leavevmode{[}numpy.ndarray{]}
The frequency vector over which the metrics have been calculated

\end{description}
\index{output\_csv() (pyzones.Metrics method)@\spxentry{output\_csv()}\spxextra{pyzones.Metrics method}}

\begin{fulllineitems}
\phantomsection\label{\detokenize{source/pyzones:pyzones.Metrics.output_csv}}\pysiglinewithargsret{\sphinxbfcode{\sphinxupquote{output\_csv}}}{\emph{file\_name}, \emph{overwrite=False}, \emph{contrast=False}, \emph{effort=False}, \emph{planarity=False}, \emph{reproduction\_error=False}}{}
Save the chosen metrics to a csv file
\begin{quote}\begin{description}
\item[{Parameters}] \leavevmode\begin{itemize}
\item {} 
\sphinxstyleliteralstrong{\sphinxupquote{file\_name}} (\sphinxstyleliteralemphasis{\sphinxupquote{str}}) \textendash{} The file path of the csv file

\item {} 
\sphinxstyleliteralstrong{\sphinxupquote{overwrite}} (\sphinxstyleliteralemphasis{\sphinxupquote{bool}}) \textendash{} Should the file be overwritten or appended to if it already exists.

\item {} 
\sphinxstyleliteralstrong{\sphinxupquote{contrast}} (\sphinxstyleliteralemphasis{\sphinxupquote{bool}}) \textendash{} If true save the acoustic contrast between the dark and bright zone to the csv

\item {} 
\sphinxstyleliteralstrong{\sphinxupquote{effort}} (\sphinxstyleliteralemphasis{\sphinxupquote{bool}}) \textendash{} If true save the difference in effort between q\_ref and the calculated filter weights to the csv

\item {} 
\sphinxstyleliteralstrong{\sphinxupquote{planarity}} (\sphinxstyleliteralemphasis{\sphinxupquote{bool}}) \textendash{} If true save the planarity of the soundfield in the bright zone to the csv

\item {} 
\sphinxstyleliteralstrong{\sphinxupquote{reproduction\_error}} (\sphinxstyleliteralemphasis{\sphinxupquote{bool}}) \textendash{} If true save the reproduction error in the bright zone to the csv

\end{itemize}

\end{description}\end{quote}

\end{fulllineitems}

\index{plot() (pyzones.Metrics method)@\spxentry{plot()}\spxextra{pyzones.Metrics method}}

\begin{fulllineitems}
\phantomsection\label{\detokenize{source/pyzones:pyzones.Metrics.plot}}\pysiglinewithargsret{\sphinxbfcode{\sphinxupquote{plot}}}{\emph{graph\_name}, \emph{file\_name}, \emph{metric}}{}
Plot the chosen metric over frequency
\begin{quote}\begin{description}
\item[{Parameters}] \leavevmode\begin{itemize}
\item {} 
\sphinxstyleliteralstrong{\sphinxupquote{graph\_name}} (\sphinxstyleliteralemphasis{\sphinxupquote{str}}) \textendash{} The name of the graph

\item {} 
\sphinxstyleliteralstrong{\sphinxupquote{file\_name}} (\sphinxstyleliteralemphasis{\sphinxupquote{str}}) \textendash{} The file path of the image made

\item {} 
\sphinxstyleliteralstrong{\sphinxupquote{metric}} (\sphinxstyleliteralemphasis{\sphinxupquote{str}}) \textendash{} The metric chosen - ‘contrast’, ‘effort’, ‘planarity’ or ‘reproduction error’

\end{itemize}

\end{description}\end{quote}

\end{fulllineitems}

\index{print() (pyzones.Metrics method)@\spxentry{print()}\spxextra{pyzones.Metrics method}}

\begin{fulllineitems}
\phantomsection\label{\detokenize{source/pyzones:pyzones.Metrics.print}}\pysiglinewithargsret{\sphinxbfcode{\sphinxupquote{print}}}{\emph{contrast=False}, \emph{effort=False}, \emph{planarity=False}, \emph{reproduction\_error=False}}{}
Prints the metrics chosen to the terminal
\begin{quote}\begin{description}
\item[{Parameters}] \leavevmode\begin{itemize}
\item {} 
\sphinxstyleliteralstrong{\sphinxupquote{contrast}} (\sphinxstyleliteralemphasis{\sphinxupquote{bool}}) \textendash{} If true print the acoustic contrast between the dark and bright zone

\item {} 
\sphinxstyleliteralstrong{\sphinxupquote{effort}} (\sphinxstyleliteralemphasis{\sphinxupquote{bool}}) \textendash{} If true print the difference in effort between q\_ref and the calculated filter weights

\item {} 
\sphinxstyleliteralstrong{\sphinxupquote{planarity}} (\sphinxstyleliteralemphasis{\sphinxupquote{bool}}) \textendash{} If true print the planarity of the soundfield in the bright zone

\item {} 
\sphinxstyleliteralstrong{\sphinxupquote{reproduction\_error}} (\sphinxstyleliteralemphasis{\sphinxupquote{bool}}) \textendash{} If true print the reproduction error in the bright zone

\end{itemize}

\end{description}\end{quote}

\end{fulllineitems}


\end{fulllineitems}

\index{Microphone (class in pyzones)@\spxentry{Microphone}\spxextra{class in pyzones}}

\begin{fulllineitems}
\phantomsection\label{\detokenize{source/pyzones:pyzones.Microphone}}\pysiglinewithargsret{\sphinxbfcode{\sphinxupquote{class }}\sphinxcode{\sphinxupquote{pyzones.}}\sphinxbfcode{\sphinxupquote{Microphone}}}{\emph{zone='none'}, \emph{purpose='none'}, \emph{position=None}, \emph{colour=None}}{}
Bases: {\hyperref[\detokenize{source/pyzones:pyzones.SoundObject}]{\sphinxcrossref{\sphinxcode{\sphinxupquote{pyzones.SoundObject}}}}}

A microphone to be used in simulations of sound zones. Inherits from the sound object class.
\begin{description}
\item[{\_radius}] \leavevmode{[}static float{]}
The radius of circles used to represent the microphones when rendered in the soundfield

\end{description}
\begin{description}
\item[{colour}] \leavevmode{[}list{]}
A list of float values (r, g, b)

\item[{position}] \leavevmode{[}list{]}
A list of coordinates (x, y) describing the position of the microphone

\item[{zone}] \leavevmode{[}str{]}
The zone in which this microphone is situated, “bright”, “dark” or “either”

\item[{purpose}] \leavevmode{[}str{]}
The purpose of the microphone, “setup”, “evaluation” or “either

\item[{\_pressure}] \leavevmode{[}list{]}
The pressure for each frequency at the microphone most recently calculated and set

\end{description}
\index{pressure (pyzones.Microphone attribute)@\spxentry{pressure}\spxextra{pyzones.Microphone attribute}}

\begin{fulllineitems}
\phantomsection\label{\detokenize{source/pyzones:pyzones.Microphone.pressure}}\pysigline{\sphinxbfcode{\sphinxupquote{pressure}}}
The pressure for each frequency at the microphone most recently calculated and set
\begin{quote}\begin{description}
\item[{Returns}] \leavevmode
The pressure

\item[{Return type}] \leavevmode
list

\end{description}\end{quote}

\end{fulllineitems}

\index{purpose (pyzones.Microphone attribute)@\spxentry{purpose}\spxextra{pyzones.Microphone attribute}}

\begin{fulllineitems}
\phantomsection\label{\detokenize{source/pyzones:pyzones.Microphone.purpose}}\pysigline{\sphinxbfcode{\sphinxupquote{purpose}}}
The purpose of the microphone, “setup” or “evaluation”
\begin{quote}\begin{description}
\item[{Returns}] \leavevmode
The purpose

\item[{Return type}] \leavevmode
str

\end{description}\end{quote}

\end{fulllineitems}

\index{radius (pyzones.Microphone attribute)@\spxentry{radius}\spxextra{pyzones.Microphone attribute}}

\begin{fulllineitems}
\phantomsection\label{\detokenize{source/pyzones:pyzones.Microphone.radius}}\pysigline{\sphinxbfcode{\sphinxupquote{radius}}}
The radius of circles used to represent the microphones when rendered in the soundfield
\begin{quote}\begin{description}
\item[{Returns}] \leavevmode
The radius

\item[{Return type}] \leavevmode
float

\end{description}\end{quote}

\end{fulllineitems}

\index{zone (pyzones.Microphone attribute)@\spxentry{zone}\spxextra{pyzones.Microphone attribute}}

\begin{fulllineitems}
\phantomsection\label{\detokenize{source/pyzones:pyzones.Microphone.zone}}\pysigline{\sphinxbfcode{\sphinxupquote{zone}}}
The zone in which this microphone is situated, “bright” or “dark”
\begin{quote}\begin{description}
\item[{Returns}] \leavevmode
The zone

\item[{Return type}] \leavevmode
str

\end{description}\end{quote}

\end{fulllineitems}


\end{fulllineitems}

\index{MicrophoneArray (class in pyzones)@\spxentry{MicrophoneArray}\spxextra{class in pyzones}}

\begin{fulllineitems}
\phantomsection\label{\detokenize{source/pyzones:pyzones.MicrophoneArray}}\pysiglinewithargsret{\sphinxbfcode{\sphinxupquote{class }}\sphinxcode{\sphinxupquote{pyzones.}}\sphinxbfcode{\sphinxupquote{MicrophoneArray}}}{\emph{*args}}{}
Bases: {\hyperref[\detokenize{source/pyzones:pyzones.SoundObjectArray}]{\sphinxcrossref{\sphinxcode{\sphinxupquote{pyzones.SoundObjectArray}}}}}

A container class for microphones
\index{get\_pressures() (pyzones.MicrophoneArray method)@\spxentry{get\_pressures()}\spxextra{pyzones.MicrophoneArray method}}

\begin{fulllineitems}
\phantomsection\label{\detokenize{source/pyzones:pyzones.MicrophoneArray.get_pressures}}\pysiglinewithargsret{\sphinxbfcode{\sphinxupquote{get\_pressures}}}{\emph{frequency\_index}}{}
Returns the pressure values for the microphone array at the frequency index
\begin{quote}\begin{description}
\item[{Parameters}] \leavevmode
\sphinxstyleliteralstrong{\sphinxupquote{frequency\_index}} (\sphinxstyleliteralemphasis{\sphinxupquote{int}}) \textendash{} The index at which the relevant frequency is stored in the simulation’s frequency vector

\item[{Returns}] \leavevmode
The list of pressures for the relevant frequency

\item[{Return type}] \leavevmode
list

\end{description}\end{quote}

\end{fulllineitems}

\index{get\_subset() (pyzones.MicrophoneArray method)@\spxentry{get\_subset()}\spxextra{pyzones.MicrophoneArray method}}

\begin{fulllineitems}
\phantomsection\label{\detokenize{source/pyzones:pyzones.MicrophoneArray.get_subset}}\pysiglinewithargsret{\sphinxbfcode{\sphinxupquote{get\_subset}}}{\emph{zone='either'}, \emph{purpose='either'}}{}
Returns a subset of microphones from the array with the required properties, “bright”, “dark” or “either” zone
and “setup”, “evaluation” or “either” purpose.
\begin{quote}\begin{description}
\item[{Parameters}] \leavevmode\begin{itemize}
\item {} 
\sphinxstyleliteralstrong{\sphinxupquote{zone}} (\sphinxstyleliteralemphasis{\sphinxupquote{str}}) \textendash{} The zone in which the subset of microphones should be positioned - “bright”, “dark” or “either”

\item {} 
\sphinxstyleliteralstrong{\sphinxupquote{purpose}} (\sphinxstyleliteralemphasis{\sphinxupquote{str}}) \textendash{} The purpose of the subset of microphones - “setup”, “evaluation” or “either”

\end{itemize}

\item[{Returns}] \leavevmode
A microphone array containing microphones of the specified requirements

\item[{Return type}] \leavevmode
{\hyperref[\detokenize{source/pyzones:pyzones.MicrophoneArray}]{\sphinxcrossref{MicrophoneArray}}}

\end{description}\end{quote}

\end{fulllineitems}

\index{initialise\_pressures() (pyzones.MicrophoneArray method)@\spxentry{initialise\_pressures()}\spxextra{pyzones.MicrophoneArray method}}

\begin{fulllineitems}
\phantomsection\label{\detokenize{source/pyzones:pyzones.MicrophoneArray.initialise_pressures}}\pysiglinewithargsret{\sphinxbfcode{\sphinxupquote{initialise\_pressures}}}{\emph{num\_frequencies}}{}
Initialise the pressures of the microphones in the microphone array to zero.
\begin{quote}\begin{description}
\item[{Parameters}] \leavevmode
\sphinxstyleliteralstrong{\sphinxupquote{num\_frequencies}} (\sphinxstyleliteralemphasis{\sphinxupquote{int}}) \textendash{} The number of frequencies in the simulation.

\end{description}\end{quote}

\end{fulllineitems}

\index{set\_pressures() (pyzones.MicrophoneArray method)@\spxentry{set\_pressures()}\spxextra{pyzones.MicrophoneArray method}}

\begin{fulllineitems}
\phantomsection\label{\detokenize{source/pyzones:pyzones.MicrophoneArray.set_pressures}}\pysiglinewithargsret{\sphinxbfcode{\sphinxupquote{set\_pressures}}}{\emph{new\_pressures}, \emph{frequency\_index}}{}
Set the pressures values for the microphone array at the frequency index
\begin{quote}\begin{description}
\item[{Parameters}] \leavevmode\begin{itemize}
\item {} 
\sphinxstyleliteralstrong{\sphinxupquote{new\_pressures}} (\sphinxstyleliteralemphasis{\sphinxupquote{list}}) \textendash{} The list of pressures

\item {} 
\sphinxstyleliteralstrong{\sphinxupquote{frequency\_index}} (\sphinxstyleliteralemphasis{\sphinxupquote{int}}) \textendash{} The index at which the relevant frequency is stored in the simulation’s frequency vector

\end{itemize}

\end{description}\end{quote}

\end{fulllineitems}


\end{fulllineitems}

\index{Rectangle (class in pyzones)@\spxentry{Rectangle}\spxextra{class in pyzones}}

\begin{fulllineitems}
\phantomsection\label{\detokenize{source/pyzones:pyzones.Rectangle}}\pysiglinewithargsret{\sphinxbfcode{\sphinxupquote{class }}\sphinxcode{\sphinxupquote{pyzones.}}\sphinxbfcode{\sphinxupquote{Rectangle}}}{\emph{coordinate}, \emph{width}, \emph{height}, \emph{coordinate\_pos='bottom left'}}{}
Bases: {\hyperref[\detokenize{source/pyzones:pyzones.Shape}]{\sphinxcrossref{\sphinxcode{\sphinxupquote{pyzones.Shape}}}}}

A geometric class for a rectangle
\begin{description}
\item[{coordinate}] \leavevmode{[}list{]}
A list of coordinates (x, y) describing the centre or bottom left of the rectangle

\item[{width}] \leavevmode{[}float{]}
The width of the rectangle

\item[{height}] \leavevmode{[}float{]}
The height of the rectangle

\item[{coordinate\_pos}] \leavevmode{[}str{]}
Describes the position of the coordinate parameter - either “centre” or “bottom left”

\end{description}
\index{get\_grid() (pyzones.Rectangle method)@\spxentry{get\_grid()}\spxextra{pyzones.Rectangle method}}

\begin{fulllineitems}
\phantomsection\label{\detokenize{source/pyzones:pyzones.Rectangle.get_grid}}\pysiglinewithargsret{\sphinxbfcode{\sphinxupquote{get\_grid}}}{\emph{spacing}}{}
Create a grid of points spaced uniformly across the rectangle
\begin{quote}\begin{description}
\item[{Parameters}] \leavevmode
\sphinxstyleliteralstrong{\sphinxupquote{spacing}} (\sphinxstyleliteralemphasis{\sphinxupquote{float}}) \textendash{} Approximate spacing between points in the grid

\item[{Returns}] \leavevmode
A list of points (x,y) uniformly spaced across the rectangle

\item[{Return type}] \leavevmode
numpy.ndarray

\end{description}\end{quote}

\end{fulllineitems}

\index{get\_perimeter() (pyzones.Rectangle method)@\spxentry{get\_perimeter()}\spxextra{pyzones.Rectangle method}}

\begin{fulllineitems}
\phantomsection\label{\detokenize{source/pyzones:pyzones.Rectangle.get_perimeter}}\pysiglinewithargsret{\sphinxbfcode{\sphinxupquote{get\_perimeter}}}{\emph{start\_point}, \emph{end\_point}, \emph{num\_points}}{}
Create a list of points between the user defined start and end positions on the perimeter of the shape
\begin{quote}\begin{description}
\item[{Parameters}] \leavevmode\begin{itemize}
\item {} 
\sphinxstyleliteralstrong{\sphinxupquote{start}} (\sphinxstyleliteralemphasis{\sphinxupquote{float}}) \textendash{} Position at which the list of points should begin

\item {} 
\sphinxstyleliteralstrong{\sphinxupquote{end}} (\sphinxstyleliteralemphasis{\sphinxupquote{float}}) \textendash{} Position at which the list of points should end

\item {} 
\sphinxstyleliteralstrong{\sphinxupquote{num\_points}} (\sphinxstyleliteralemphasis{\sphinxupquote{int}}) \textendash{} Number of points

\end{itemize}

\item[{Returns}] \leavevmode
A list of points (x,y) evenly spaced on the perimeter of the shape between the start and end positions

\item[{Return type}] \leavevmode
numpy.ndarray

\end{description}\end{quote}

\end{fulllineitems}

\index{height (pyzones.Rectangle attribute)@\spxentry{height}\spxextra{pyzones.Rectangle attribute}}

\begin{fulllineitems}
\phantomsection\label{\detokenize{source/pyzones:pyzones.Rectangle.height}}\pysigline{\sphinxbfcode{\sphinxupquote{height}}}
The height of the rectangle
\begin{quote}\begin{description}
\item[{Returns}] \leavevmode
The height

\item[{Return type}] \leavevmode
float

\end{description}\end{quote}

\end{fulllineitems}

\index{is\_point\_inside() (pyzones.Rectangle method)@\spxentry{is\_point\_inside()}\spxextra{pyzones.Rectangle method}}

\begin{fulllineitems}
\phantomsection\label{\detokenize{source/pyzones:pyzones.Rectangle.is_point_inside}}\pysiglinewithargsret{\sphinxbfcode{\sphinxupquote{is\_point\_inside}}}{\emph{point}}{}
Check whether or not a point is inside the rectangle
\begin{quote}\begin{description}
\item[{Parameters}] \leavevmode
\sphinxstyleliteralstrong{\sphinxupquote{point}} (\sphinxstyleliteralemphasis{\sphinxupquote{tuple}}) \textendash{} list/tuple of the coordinates (x, y) of a point

\item[{Returns}] \leavevmode
A bool stating whether or not the point is within the rectangle

\item[{Return type}] \leavevmode
bool

\end{description}\end{quote}

\end{fulllineitems}

\index{width (pyzones.Rectangle attribute)@\spxentry{width}\spxextra{pyzones.Rectangle attribute}}

\begin{fulllineitems}
\phantomsection\label{\detokenize{source/pyzones:pyzones.Rectangle.width}}\pysigline{\sphinxbfcode{\sphinxupquote{width}}}
The width of the rectangle
\begin{quote}\begin{description}
\item[{Returns}] \leavevmode
The width

\item[{Return type}] \leavevmode
float

\end{description}\end{quote}

\end{fulllineitems}

\index{xy (pyzones.Rectangle attribute)@\spxentry{xy}\spxextra{pyzones.Rectangle attribute}}

\begin{fulllineitems}
\phantomsection\label{\detokenize{source/pyzones:pyzones.Rectangle.xy}}\pysigline{\sphinxbfcode{\sphinxupquote{xy}}}
A list of coordinates (x, y) describing the bottom left of the rectangle
\begin{quote}\begin{description}
\item[{Returns}] \leavevmode
(x, y) of the bottom left of the rectangle

\item[{Return type}] \leavevmode
list

\end{description}\end{quote}

\end{fulllineitems}


\end{fulllineitems}

\index{Shape (class in pyzones)@\spxentry{Shape}\spxextra{class in pyzones}}

\begin{fulllineitems}
\phantomsection\label{\detokenize{source/pyzones:pyzones.Shape}}\pysigline{\sphinxbfcode{\sphinxupquote{class }}\sphinxcode{\sphinxupquote{pyzones.}}\sphinxbfcode{\sphinxupquote{Shape}}}
Bases: \sphinxcode{\sphinxupquote{object}}

An abstract class for geometric shapes defining some key methods required
\index{get\_grid() (pyzones.Shape method)@\spxentry{get\_grid()}\spxextra{pyzones.Shape method}}

\begin{fulllineitems}
\phantomsection\label{\detokenize{source/pyzones:pyzones.Shape.get_grid}}\pysiglinewithargsret{\sphinxbfcode{\sphinxupquote{get\_grid}}}{\emph{spacing}}{}
Create a grid of points spaced uniformly across the shape
\begin{quote}\begin{description}
\item[{Parameters}] \leavevmode
\sphinxstyleliteralstrong{\sphinxupquote{spacing}} (\sphinxstyleliteralemphasis{\sphinxupquote{float}}) \textendash{} Spacing between points in the grid

\item[{Returns}] \leavevmode
A list of points (x,y) uniformly space across the shape

\item[{Return type}] \leavevmode
numpy.ndarray

\end{description}\end{quote}

\end{fulllineitems}

\index{get\_perimeter() (pyzones.Shape method)@\spxentry{get\_perimeter()}\spxextra{pyzones.Shape method}}

\begin{fulllineitems}
\phantomsection\label{\detokenize{source/pyzones:pyzones.Shape.get_perimeter}}\pysiglinewithargsret{\sphinxbfcode{\sphinxupquote{get\_perimeter}}}{\emph{start}, \emph{end}, \emph{num\_points}}{}
Create a list of points between the user defined start and end positions on the perimeter of the shape
\begin{quote}\begin{description}
\item[{Parameters}] \leavevmode\begin{itemize}
\item {} 
\sphinxstyleliteralstrong{\sphinxupquote{start}} (\sphinxstyleliteralemphasis{\sphinxupquote{float}}) \textendash{} Position at which the list of points should begin

\item {} 
\sphinxstyleliteralstrong{\sphinxupquote{end}} (\sphinxstyleliteralemphasis{\sphinxupquote{float}}) \textendash{} Position at which the list of points should end

\item {} 
\sphinxstyleliteralstrong{\sphinxupquote{num\_points}} (\sphinxstyleliteralemphasis{\sphinxupquote{int}}) \textendash{} Number of points

\end{itemize}

\item[{Returns}] \leavevmode
A list of points (x,y) evenly spaced on the perimeter of the shape between the start and end positions

\item[{Return type}] \leavevmode
numpy.ndarray

\end{description}\end{quote}

\end{fulllineitems}

\index{is\_point\_inside() (pyzones.Shape method)@\spxentry{is\_point\_inside()}\spxextra{pyzones.Shape method}}

\begin{fulllineitems}
\phantomsection\label{\detokenize{source/pyzones:pyzones.Shape.is_point_inside}}\pysiglinewithargsret{\sphinxbfcode{\sphinxupquote{is\_point\_inside}}}{\emph{point}}{}
Check whether or not a point is inside the shape
\begin{quote}\begin{description}
\item[{Parameters}] \leavevmode
\sphinxstyleliteralstrong{\sphinxupquote{point}} (\sphinxstyleliteralemphasis{\sphinxupquote{list}}) \textendash{} list/tuple of the coordinates (x, y) of a point

\item[{Returns}] \leavevmode
A bool stating whether or not the point is within the shape

\item[{Return type}] \leavevmode
bool

\end{description}\end{quote}

\end{fulllineitems}


\end{fulllineitems}

\index{Simulation (class in pyzones)@\spxentry{Simulation}\spxextra{class in pyzones}}

\begin{fulllineitems}
\phantomsection\label{\detokenize{source/pyzones:pyzones.Simulation}}\pysiglinewithargsret{\sphinxbfcode{\sphinxupquote{class }}\sphinxcode{\sphinxupquote{pyzones.}}\sphinxbfcode{\sphinxupquote{Simulation}}}{\emph{frequencies}, \emph{ls\_array}, \emph{mic\_array}, \emph{c=343}, \emph{rho=1.21}, \emph{target\_spl=76}, \emph{planarity\_constants=(360}, \emph{360}, \emph{5)}, \emph{pm\_angle=0}}{}
Bases: \sphinxcode{\sphinxupquote{object}}

Where all of the calculations for the simulation happen. Using the input of the different array and zone
geometries calculates filter weights across a range of frequencies for different methods of sound zone optimisation
\begin{itemize}
\item {} 
Brightness Control, Acoustic Contrast Control, Planarity Control and Pressure Matching.

\end{itemize}
\begin{description}
\item[{frequencies}] \leavevmode{[}numpy.ndarray{]}
vector of frequencies

\item[{\_omega}] \leavevmode{[}numpy.ndarray{]}
vector of angular frequencies

\item[{\_k}] \leavevmode{[}numpy.array{]}
vector of wave numbers

\item[{\_c}] \leavevmode{[}float{]}
the speed of sound

\item[{\_rho}] \leavevmode{[}float{]}
the density of air

\item[{\_target\_spl}] \leavevmode{[}float{]}
The target SPL in dB in the bright zone

\item[{\_target\_pa}] \leavevmode{[}float{]}
The target pressure in the bright zone

\item[{ls\_array}] \leavevmode{[}LoudspeakerArray{]}
The loudspeaker array used in the simulation

\item[{mic\_array}] \leavevmode{[}MicrophoneArray{]}
The microphone array used in the simulation

\item[{\_tf\_array\_ls}] \leavevmode{[}numpy.ndarray{]}
The transfer functions with the loudspeakers in the first axis

\item[{\_tf\_array\_mic}] \leavevmode{[}numpy.ndarray{]}
The transfer functions with the microphones in the first axis

\item[{\_q\_ref}] \leavevmode{[}float{]}
The reference filter weight for a loudspeaker to realise the target SPL. Used in effort calculations

\item[{\_current\_method}] \leavevmode{[}str{]}
A string of the current method, i.e the most recently calculated filter weights and metrics refer to this method

\item[{steering\_vectors}] \leavevmode{[}list{]}
A list of steering vectors to contain the setup and evaluation steering vectors for the bright zone

\item[{\_planarity\_constants}] \leavevmode{[}tuple{]}
A tuple containing three constants for Planarity Control. (start\_angle, end\_angle, transition\_angle)

\item[{\_pm\_angle}] \leavevmode{[}float{]}
The angle of incidence for the Pressure Matching method.

\end{description}
\index{add\_steering\_vectors() (pyzones.Simulation method)@\spxentry{add\_steering\_vectors()}\spxextra{pyzones.Simulation method}}

\begin{fulllineitems}
\phantomsection\label{\detokenize{source/pyzones:pyzones.Simulation.add_steering_vectors}}\pysiglinewithargsret{\sphinxbfcode{\sphinxupquote{add\_steering\_vectors}}}{\emph{steering\_vectors}}{}
Add planarity steering vectors to the simulation for use in calculating Planarity Control filter weights and
planarity evaluation metrics.
\begin{quote}\begin{description}
\item[{Parameters}] \leavevmode
\sphinxstyleliteralstrong{\sphinxupquote{steering\_vectors}} (\sphinxstyleliteralemphasis{\sphinxupquote{list}}) \textendash{} The input steering vector(s)

\end{description}\end{quote}

\end{fulllineitems}

\index{c (pyzones.Simulation attribute)@\spxentry{c}\spxextra{pyzones.Simulation attribute}}

\begin{fulllineitems}
\phantomsection\label{\detokenize{source/pyzones:pyzones.Simulation.c}}\pysigline{\sphinxbfcode{\sphinxupquote{c}}}
The speed of sound
\begin{quote}\begin{description}
\item[{Returns}] \leavevmode
the speed of sound

\item[{Return type}] \leavevmode
float

\end{description}\end{quote}

\end{fulllineitems}

\index{calculate\_filter\_weights() (pyzones.Simulation method)@\spxentry{calculate\_filter\_weights()}\spxextra{pyzones.Simulation method}}

\begin{fulllineitems}
\phantomsection\label{\detokenize{source/pyzones:pyzones.Simulation.calculate_filter_weights}}\pysiglinewithargsret{\sphinxbfcode{\sphinxupquote{calculate\_filter\_weights}}}{\emph{method='BC'}}{}
Calculate filter weights for the loudspeaker array based on the transfer functions of the microphones in the mic
array with the purpose of “bright” or “either”. Four different methods are available - Brightness control ‘BC’,
Acoustic Contrast Control ‘ACC’, Planarity Control ‘PC’ and Pressure Matching ‘PM’. These filter weights are not
returned but automatically assigned to the loudspeaker array.
\begin{quote}\begin{description}
\item[{Parameters}] \leavevmode
\sphinxstyleliteralstrong{\sphinxupquote{method}} (\sphinxstyleliteralemphasis{\sphinxupquote{str}}) \textendash{} The string containing the method for which the filter weights should be calculated.

\end{description}\end{quote}

\end{fulllineitems}

\index{calculate\_metrics() (pyzones.Simulation method)@\spxentry{calculate\_metrics()}\spxextra{pyzones.Simulation method}}

\begin{fulllineitems}
\phantomsection\label{\detokenize{source/pyzones:pyzones.Simulation.calculate_metrics}}\pysiglinewithargsret{\sphinxbfcode{\sphinxupquote{calculate\_metrics}}}{\emph{contrast=False}, \emph{effort=False}, \emph{planarity=False}, \emph{reproduction\_error=False}}{}
Calculate the metrics for the most recently calculated filter weights. These metrics can be chosen using the
optional arguments. Contrast, effort, planarity and reproduction\_error. Be advised ther reproduction error is only
appropriate with pressure matching method.
\begin{quote}\begin{description}
\item[{Parameters}] \leavevmode\begin{itemize}
\item {} 
\sphinxstyleliteralstrong{\sphinxupquote{contrast}} (\sphinxstyleliteralemphasis{\sphinxupquote{bool}}) \textendash{} If true calculate the acoustic contrast between the dark and bright zone

\item {} 
\sphinxstyleliteralstrong{\sphinxupquote{effort}} (\sphinxstyleliteralemphasis{\sphinxupquote{bool}}) \textendash{} If true calculate the difference in effort between q\_ref and the calculated filter weights

\item {} 
\sphinxstyleliteralstrong{\sphinxupquote{planarity}} (\sphinxstyleliteralemphasis{\sphinxupquote{bool}}) \textendash{} If true calculate the planarity of the soundfield in the bright zone

\item {} 
\sphinxstyleliteralstrong{\sphinxupquote{reproduction\_error}} (\sphinxstyleliteralemphasis{\sphinxupquote{bool}}) \textendash{} If true calculate the reproduction error in the bright zone

\end{itemize}

\item[{Returns}] \leavevmode
The calculated metrics in the Metrics container class

\end{description}\end{quote}

\end{fulllineitems}

\index{calculate\_sound\_pressures() (pyzones.Simulation method)@\spxentry{calculate\_sound\_pressures()}\spxextra{pyzones.Simulation method}}

\begin{fulllineitems}
\phantomsection\label{\detokenize{source/pyzones:pyzones.Simulation.calculate_sound_pressures}}\pysiglinewithargsret{\sphinxbfcode{\sphinxupquote{calculate\_sound\_pressures}}}{\emph{tf\_array}, \emph{mic\_array}}{}
Calculate the sound pressure at each microphone across each frequency for the most recent filter weights. Sets
the pressures of the microphones in mic\_array
\begin{quote}\begin{description}
\item[{Parameters}] \leavevmode\begin{itemize}
\item {} 
\sphinxstyleliteralstrong{\sphinxupquote{tf\_array}} (\sphinxstyleliteralemphasis{\sphinxupquote{numpy.ndarray}}) \textendash{} Transfer functions of the MicrophoneArray and LoudspeakerArray

\item {} 
\sphinxstyleliteralstrong{\sphinxupquote{mic\_array}} ({\hyperref[\detokenize{source/pyzones:pyzones.MicrophoneArray}]{\sphinxcrossref{\sphinxstyleliteralemphasis{\sphinxupquote{MicrophoneArray}}}}}) \textendash{} The MicrophoneArray used

\end{itemize}

\end{description}\end{quote}

\end{fulllineitems}

\index{calculate\_transfer\_functions() (pyzones.Simulation method)@\spxentry{calculate\_transfer\_functions()}\spxextra{pyzones.Simulation method}}

\begin{fulllineitems}
\phantomsection\label{\detokenize{source/pyzones:pyzones.Simulation.calculate_transfer_functions}}\pysiglinewithargsret{\sphinxbfcode{\sphinxupquote{calculate\_transfer\_functions}}}{\emph{orientation}, \emph{mic\_array=None}, \emph{frequency=None}}{}
Calculate transfer functions with the given orientation. If no mic\_array or frequencies are provided the
transfer functions returned are those of the mic\_array and frequency initialised when creating the simulation.
Be advised that the transfer functions for the simulation are already made when you create a simulation object.
This should only need to be used for external calculations hence why the optional parameters are provided.
\begin{quote}\begin{description}
\item[{Parameters}] \leavevmode\begin{itemize}
\item {} 
\sphinxstyleliteralstrong{\sphinxupquote{orientation}} (\sphinxstyleliteralemphasis{\sphinxupquote{str}}) \textendash{} String to choose the orientation of transfer functions - “microphones” or “loudspeakers”

\item {} 
\sphinxstyleliteralstrong{\sphinxupquote{mic\_array}} ({\hyperref[\detokenize{source/pyzones:pyzones.MicrophoneArray}]{\sphinxcrossref{\sphinxstyleliteralemphasis{\sphinxupquote{MicrophoneArray}}}}}) \textendash{} The MicrophoneArray to be used. Defaults to None - the simulations microphone array

\item {} 
\sphinxstyleliteralstrong{\sphinxupquote{frequency}} (\sphinxstyleliteralemphasis{\sphinxupquote{numpy.ndarray}}) \textendash{} The frequencies across which the transfer functions are calculated

\end{itemize}

\item[{Returns}] \leavevmode
A array of shape (freq, mics, ls) or (freq, ls, mic) depending on orientation

\item[{Return type}] \leavevmode
numpy.ndarray

\end{description}\end{quote}

\end{fulllineitems}

\index{get\_avg\_pa() (pyzones.Simulation method)@\spxentry{get\_avg\_pa()}\spxextra{pyzones.Simulation method}}

\begin{fulllineitems}
\phantomsection\label{\detokenize{source/pyzones:pyzones.Simulation.get_avg_pa}}\pysiglinewithargsret{\sphinxbfcode{\sphinxupquote{get\_avg\_pa}}}{\emph{frequency\_index}, \emph{purpose='either'}, \emph{zone='either'}}{}
Get the average pressure in the zone across the microphones of the given purpose. The index of the frequency in
the frequency vector must be given.
\begin{quote}\begin{description}
\item[{Parameters}] \leavevmode\begin{itemize}
\item {} 
\sphinxstyleliteralstrong{\sphinxupquote{frequency\_index}} (\sphinxstyleliteralemphasis{\sphinxupquote{int}}) \textendash{} The index of the frequency in the frequency vector

\item {} 
\sphinxstyleliteralstrong{\sphinxupquote{zone}} (\sphinxstyleliteralemphasis{\sphinxupquote{str}}) \textendash{} The zone in which the pressure should be averaged - “bright”, “dark” or “either”

\item {} 
\sphinxstyleliteralstrong{\sphinxupquote{purpose}} (\sphinxstyleliteralemphasis{\sphinxupquote{str}}) \textendash{} The purpose of microphones for which the pressure should be averaged -  “setup”, “evaluation” or
“either”

\end{itemize}

\item[{Returns}] \leavevmode
The average pressure in pascals

\item[{Return type}] \leavevmode
float

\end{description}\end{quote}

\end{fulllineitems}

\index{get\_tf\_subset() (pyzones.Simulation method)@\spxentry{get\_tf\_subset()}\spxextra{pyzones.Simulation method}}

\begin{fulllineitems}
\phantomsection\label{\detokenize{source/pyzones:pyzones.Simulation.get_tf_subset}}\pysiglinewithargsret{\sphinxbfcode{\sphinxupquote{get\_tf\_subset}}}{\emph{orientation}, \emph{zone='either'}, \emph{purpose='either'}}{}
Returns a subset of the simulation’s transfer functions with the given orientation. The subset is chosen based
on the parameters zone and purpose which can be. “bright”, “dark” or “either” and “setup”, “evaluation” or
“either” respectively. This also returns a subset MicrophoneArray of the the microphones used in the transfer
functions.
\begin{quote}\begin{description}
\item[{Parameters}] \leavevmode\begin{itemize}
\item {} 
\sphinxstyleliteralstrong{\sphinxupquote{orientation}} (\sphinxstyleliteralemphasis{\sphinxupquote{str}}) \textendash{} String to choose the orientation of transfer functions - “microphones” or “loudspeakers”

\item {} 
\sphinxstyleliteralstrong{\sphinxupquote{zone}} (\sphinxstyleliteralemphasis{\sphinxupquote{str}}) \textendash{} The zone from which the subset is taken, “bright”, “dark” or “either”

\item {} 
\sphinxstyleliteralstrong{\sphinxupquote{purpose}} (\sphinxstyleliteralemphasis{\sphinxupquote{str}}) \textendash{} The purpose which the subset has, “setup”, “evaluation” or “either”

\end{itemize}

\item[{Returns}] \leavevmode
MicrophoneArray and transfer functions of the subset

\item[{Return type}] \leavevmode
{\hyperref[\detokenize{source/pyzones:pyzones.MicrophoneArray}]{\sphinxcrossref{MicrophoneArray}}}, numpy.ndarray

\end{description}\end{quote}

\end{fulllineitems}

\index{get\_transfer\_functions() (pyzones.Simulation method)@\spxentry{get\_transfer\_functions()}\spxextra{pyzones.Simulation method}}

\begin{fulllineitems}
\phantomsection\label{\detokenize{source/pyzones:pyzones.Simulation.get_transfer_functions}}\pysiglinewithargsret{\sphinxbfcode{\sphinxupquote{get\_transfer\_functions}}}{\emph{orientation}}{}
Returns the transfer functions of the simulations with the given orientation
\begin{quote}\begin{description}
\item[{Parameters}] \leavevmode
\sphinxstyleliteralstrong{\sphinxupquote{orientation}} (\sphinxstyleliteralemphasis{\sphinxupquote{str}}) \textendash{} String to choose the orientation of transfer functions - “microphones” or “loudspeakers”

\item[{Returns}] \leavevmode
transfer functions

\item[{Return type}] \leavevmode
numpy.ndarray

\end{description}\end{quote}

\end{fulllineitems}

\index{k (pyzones.Simulation attribute)@\spxentry{k}\spxextra{pyzones.Simulation attribute}}

\begin{fulllineitems}
\phantomsection\label{\detokenize{source/pyzones:pyzones.Simulation.k}}\pysigline{\sphinxbfcode{\sphinxupquote{k}}}
vector of wave numbers
\begin{quote}\begin{description}
\item[{Returns}] \leavevmode
vector of wave numbers

\item[{Return type}] \leavevmode
numpy.ndarray

\end{description}\end{quote}

\end{fulllineitems}

\index{omega (pyzones.Simulation attribute)@\spxentry{omega}\spxextra{pyzones.Simulation attribute}}

\begin{fulllineitems}
\phantomsection\label{\detokenize{source/pyzones:pyzones.Simulation.omega}}\pysigline{\sphinxbfcode{\sphinxupquote{omega}}}
vector of angular frequencies
\begin{quote}\begin{description}
\item[{Returns}] \leavevmode
vector of angular frequencies

\item[{Return type}] \leavevmode
numpy.ndarray

\end{description}\end{quote}

\end{fulllineitems}

\index{rho (pyzones.Simulation attribute)@\spxentry{rho}\spxextra{pyzones.Simulation attribute}}

\begin{fulllineitems}
\phantomsection\label{\detokenize{source/pyzones:pyzones.Simulation.rho}}\pysigline{\sphinxbfcode{\sphinxupquote{rho}}}
The density of air
\begin{quote}\begin{description}
\item[{Returns}] \leavevmode
The density of air

\item[{Return type}] \leavevmode
float

\end{description}\end{quote}

\end{fulllineitems}

\index{target\_spl (pyzones.Simulation attribute)@\spxentry{target\_spl}\spxextra{pyzones.Simulation attribute}}

\begin{fulllineitems}
\phantomsection\label{\detokenize{source/pyzones:pyzones.Simulation.target_spl}}\pysigline{\sphinxbfcode{\sphinxupquote{target\_spl}}}
The target SPL in the bright zone
\begin{quote}\begin{description}
\item[{Returns}] \leavevmode
The target SPL in the bright zone

\item[{Return type}] \leavevmode
float

\end{description}\end{quote}

\end{fulllineitems}


\end{fulllineitems}

\index{SoundObject (class in pyzones)@\spxentry{SoundObject}\spxextra{class in pyzones}}

\begin{fulllineitems}
\phantomsection\label{\detokenize{source/pyzones:pyzones.SoundObject}}\pysiglinewithargsret{\sphinxbfcode{\sphinxupquote{class }}\sphinxcode{\sphinxupquote{pyzones.}}\sphinxbfcode{\sphinxupquote{SoundObject}}}{\emph{position=None}, \emph{colour=None}}{}
Bases: \sphinxcode{\sphinxupquote{object}}

A sound object to be used in simulations of sound zones
\begin{description}
\item[{colour}] \leavevmode{[}list{]}
A list of float values (r, g, b)

\item[{position}] \leavevmode{[}list{]}
A list of coordinates (x, y) describing the centre of the sound object

\end{description}
\index{colour (pyzones.SoundObject attribute)@\spxentry{colour}\spxextra{pyzones.SoundObject attribute}}

\begin{fulllineitems}
\phantomsection\label{\detokenize{source/pyzones:pyzones.SoundObject.colour}}\pysigline{\sphinxbfcode{\sphinxupquote{colour}}}
A list of float values (r, g, b)
\begin{quote}\begin{description}
\item[{Returns}] \leavevmode
A list of float values (r, g, b)

\item[{Return type}] \leavevmode
list

\end{description}\end{quote}

\end{fulllineitems}

\index{position (pyzones.SoundObject attribute)@\spxentry{position}\spxextra{pyzones.SoundObject attribute}}

\begin{fulllineitems}
\phantomsection\label{\detokenize{source/pyzones:pyzones.SoundObject.position}}\pysigline{\sphinxbfcode{\sphinxupquote{position}}}
A list of coordinates (x, y) describing the position of the sound object
\begin{quote}\begin{description}
\item[{Returns}] \leavevmode
(x, y) of the bottom left of the rectangle

\item[{Return type}] \leavevmode
list

\end{description}\end{quote}

\end{fulllineitems}


\end{fulllineitems}

\index{SoundObjectArray (class in pyzones)@\spxentry{SoundObjectArray}\spxextra{class in pyzones}}

\begin{fulllineitems}
\phantomsection\label{\detokenize{source/pyzones:pyzones.SoundObjectArray}}\pysiglinewithargsret{\sphinxbfcode{\sphinxupquote{class }}\sphinxcode{\sphinxupquote{pyzones.}}\sphinxbfcode{\sphinxupquote{SoundObjectArray}}}{\emph{*args}}{}
Bases: \sphinxcode{\sphinxupquote{list}}

A container class for sound objects
\index{get\_object\_positions() (pyzones.SoundObjectArray method)@\spxentry{get\_object\_positions()}\spxextra{pyzones.SoundObjectArray method}}

\begin{fulllineitems}
\phantomsection\label{\detokenize{source/pyzones:pyzones.SoundObjectArray.get_object_positions}}\pysiglinewithargsret{\sphinxbfcode{\sphinxupquote{get\_object\_positions}}}{}{}
Returns a list of the positions of the sound objects
\begin{quote}\begin{description}
\item[{Returns}] \leavevmode
list of positions

\item[{Return type}] \leavevmode
list

\end{description}\end{quote}

\end{fulllineitems}

\index{position\_objects() (pyzones.SoundObjectArray method)@\spxentry{position\_objects()}\spxextra{pyzones.SoundObjectArray method}}

\begin{fulllineitems}
\phantomsection\label{\detokenize{source/pyzones:pyzones.SoundObjectArray.position_objects}}\pysiglinewithargsret{\sphinxbfcode{\sphinxupquote{position\_objects}}}{\emph{positions}}{}
Position the sound objects
\begin{quote}\begin{description}
\item[{Parameters}] \leavevmode
\sphinxstyleliteralstrong{\sphinxupquote{positions}} (\sphinxstyleliteralemphasis{\sphinxupquote{numpy.ndarray}}) \textendash{} A list of positions the same length as the number of objects

\end{description}\end{quote}

\end{fulllineitems}


\end{fulllineitems}

\index{Soundfield (class in pyzones)@\spxentry{Soundfield}\spxextra{class in pyzones}}

\begin{fulllineitems}
\phantomsection\label{\detokenize{source/pyzones:pyzones.Soundfield}}\pysiglinewithargsret{\sphinxbfcode{\sphinxupquote{class }}\sphinxcode{\sphinxupquote{pyzones.}}\sphinxbfcode{\sphinxupquote{Soundfield}}}{\emph{coordinate}, \emph{width}, \emph{height}, \emph{coordinate\_pos='bottom left'}}{}
Bases: {\hyperref[\detokenize{source/pyzones:pyzones.Rectangle}]{\sphinxcrossref{\sphinxcode{\sphinxupquote{pyzones.Rectangle}}}}}

The soundfield being used in the simulation. Can be thought of as the room, however no room reflections are modelled
\begin{description}
\item[{\_zones}] \leavevmode{[}list{]}
A list of the zones used in the simulation.

\item[{\_fig}] \leavevmode{[}float{]}
The figure from the matplotlib.pyplot

\item[{\_axes}] \leavevmode{[}float{]}
The axes from the matplotlib.pyplot

\end{description}
\index{add\_sound\_objects() (pyzones.Soundfield method)@\spxentry{add\_sound\_objects()}\spxextra{pyzones.Soundfield method}}

\begin{fulllineitems}
\phantomsection\label{\detokenize{source/pyzones:pyzones.Soundfield.add_sound_objects}}\pysiglinewithargsret{\sphinxbfcode{\sphinxupquote{add\_sound\_objects}}}{\emph{*args}}{}
Add the sound objects to the soundfield such that they can be seen in the visualisations of the soundfield
\begin{quote}\begin{description}
\item[{Parameters}] \leavevmode
\sphinxstyleliteralstrong{\sphinxupquote{args}} \textendash{} a single Microphone/Loudspeaker or a MicrophoneArray/LoudspeakerArray

\end{description}\end{quote}

\end{fulllineitems}

\index{add\_zones() (pyzones.Soundfield method)@\spxentry{add\_zones()}\spxextra{pyzones.Soundfield method}}

\begin{fulllineitems}
\phantomsection\label{\detokenize{source/pyzones:pyzones.Soundfield.add_zones}}\pysiglinewithargsret{\sphinxbfcode{\sphinxupquote{add\_zones}}}{\emph{zones}}{}
Add the sound zone(s) to the soundfield such that they can be seen in the visualisations of the soundfield
\begin{quote}\begin{description}
\item[{Parameters}] \leavevmode
\sphinxstyleliteralstrong{\sphinxupquote{zones}} (\sphinxstyleliteralemphasis{\sphinxupquote{list}}\sphinxstyleliteralemphasis{\sphinxupquote{{[}}}{\hyperref[\detokenize{source/pyzones:pyzones.Zone}]{\sphinxcrossref{\sphinxstyleliteralemphasis{\sphinxupquote{Zone}}}}}\sphinxstyleliteralemphasis{\sphinxupquote{{]}}}) \textendash{} The zone(s) to be added to the soundfield

\end{description}\end{quote}

\end{fulllineitems}

\index{clear\_graphs() (pyzones.Soundfield method)@\spxentry{clear\_graphs()}\spxextra{pyzones.Soundfield method}}

\begin{fulllineitems}
\phantomsection\label{\detokenize{source/pyzones:pyzones.Soundfield.clear_graphs}}\pysiglinewithargsret{\sphinxbfcode{\sphinxupquote{clear\_graphs}}}{}{}
Clear the SoundObjects and Zones from the visualisations

\end{fulllineitems}

\index{plot\_geometry() (pyzones.Soundfield method)@\spxentry{plot\_geometry()}\spxextra{pyzones.Soundfield method}}

\begin{fulllineitems}
\phantomsection\label{\detokenize{source/pyzones:pyzones.Soundfield.plot_geometry}}\pysiglinewithargsret{\sphinxbfcode{\sphinxupquote{plot\_geometry}}}{\emph{graph\_name}}{}
Plot the geometry of the soundfield with any Zones or SoundObjects added
\begin{quote}\begin{description}
\item[{Parameters}] \leavevmode
\sphinxstyleliteralstrong{\sphinxupquote{graph\_name}} (\sphinxstyleliteralemphasis{\sphinxupquote{str}}) \textendash{} The name and file location of the graph

\end{description}\end{quote}

\end{fulllineitems}

\index{visualise() (pyzones.Soundfield method)@\spxentry{visualise()}\spxextra{pyzones.Soundfield method}}

\begin{fulllineitems}
\phantomsection\label{\detokenize{source/pyzones:pyzones.Soundfield.visualise}}\pysiglinewithargsret{\sphinxbfcode{\sphinxupquote{visualise}}}{\emph{sim}, \emph{graph\_name}, \emph{frequency=500}, \emph{sf\_spacing=0.1}, \emph{zone\_spacing=0.05}, \emph{zone\_alpha=2}, \emph{transfer\_functions=None}, \emph{grid=None}}{}
Create a visualisation of the Soundfield at the given frequency. The frequency chosen must have been present in
the simulation provided. Transfer functions and visualisation micr positions can be provided to prevent them
being calculated more than once. Should the same frequency, loudspeakers and visualisation microphone positions
be kept the same, the returned transfer functions and microphone positions can be used again. The filter weights
used will be those most recently calculated in the simulation.
\begin{quote}\begin{description}
\item[{Parameters}] \leavevmode\begin{itemize}
\item {} 
\sphinxstyleliteralstrong{\sphinxupquote{sim}} ({\hyperref[\detokenize{source/pyzones:pyzones.Simulation}]{\sphinxcrossref{\sphinxstyleliteralemphasis{\sphinxupquote{Simulation}}}}}) \textendash{} Simulation for which the visualation is made - contains the filter weights.

\item {} 
\sphinxstyleliteralstrong{\sphinxupquote{graph\_name}} (\sphinxstyleliteralemphasis{\sphinxupquote{str}}) \textendash{} The name and file location of the graph

\item {} 
\sphinxstyleliteralstrong{\sphinxupquote{frequency}} (\sphinxstyleliteralemphasis{\sphinxupquote{int}}) \textendash{} Frequency at which the visualisation should be made - must have been present in the Simulation

\item {} 
\sphinxstyleliteralstrong{\sphinxupquote{sf\_spacing}} (\sphinxstyleliteralemphasis{\sphinxupquote{float}}) \textendash{} The spacing between the microphones in the grid across the soundfield in metres

\item {} 
\sphinxstyleliteralstrong{\sphinxupquote{zone\_spacing}} (\sphinxstyleliteralemphasis{\sphinxupquote{float}}) \textendash{} The spacing between the microphones in the grid in the zone in metres

\item {} 
\sphinxstyleliteralstrong{\sphinxupquote{zone\_alpha}} (\sphinxstyleliteralemphasis{\sphinxupquote{float}}) \textendash{} Determines the evenness of the boundary - 0 is jagged, 2 is smooth. Above 2 is not recommended

\item {} 
\sphinxstyleliteralstrong{\sphinxupquote{transfer\_functions}} (\sphinxstyleliteralemphasis{\sphinxupquote{numpy.ndarray}}) \textendash{} ndarray of transfer functions shape (microphones, loudspeakers). None - calculated

\item {} 
\sphinxstyleliteralstrong{\sphinxupquote{grid}} (\sphinxstyleliteralemphasis{\sphinxupquote{list}}) \textendash{} Grid of mic positions (x, y) matching up with the provided transfer function. None - calculated

\end{itemize}

\item[{Returns}] \leavevmode
The grid and transfer functions used in the visualisation to prevent their unnecessary recalculation.

\item[{Return type}] \leavevmode
numpy.ndarray, list

\end{description}\end{quote}

\end{fulllineitems}


\end{fulllineitems}

\index{SteeringVector (class in pyzones)@\spxentry{SteeringVector}\spxextra{class in pyzones}}

\begin{fulllineitems}
\phantomsection\label{\detokenize{source/pyzones:pyzones.SteeringVector}}\pysiglinewithargsret{\sphinxbfcode{\sphinxupquote{class }}\sphinxcode{\sphinxupquote{pyzones.}}\sphinxbfcode{\sphinxupquote{SteeringVector}}}{\emph{sim}, \emph{purpose}, \emph{pass\_beam=3}, \emph{stop\_beam=6}, \emph{nTheta=360}, \emph{beta=0.001}, \emph{file\_path=None}}{}
Bases: \sphinxcode{\sphinxupquote{object}}

Steering vectors are used in planarity control and evaluation.
\begin{description}
\item[{hTheta}] \leavevmode{[}numpy.ndarray{]}
The first component of the steering vector

\item[{hha}] \leavevmode{[}numpy.ndarray{]}
The second component of the steering vector

\item[{nTheta}] \leavevmode{[}int{]}
The number of angles

\item[{purpose}] \leavevmode{[}str{]}
The purpose of the steering vector. “setup” for Planarity Control or “evaluation” for the planarity metric.

\end{description}
\index{save() (pyzones.SteeringVector method)@\spxentry{save()}\spxextra{pyzones.SteeringVector method}}

\begin{fulllineitems}
\phantomsection\label{\detokenize{source/pyzones:pyzones.SteeringVector.save}}\pysiglinewithargsret{\sphinxbfcode{\sphinxupquote{save}}}{\emph{file\_path}}{}
Save a steering vector to the given file path
\begin{quote}\begin{description}
\item[{Parameters}] \leavevmode
\sphinxstyleliteralstrong{\sphinxupquote{file\_path}} (\sphinxstyleliteralemphasis{\sphinxupquote{str}}) \textendash{} The file path at which the steering vector will be saved

\end{description}\end{quote}

\end{fulllineitems}


\end{fulllineitems}

\index{Zone (class in pyzones)@\spxentry{Zone}\spxextra{class in pyzones}}

\begin{fulllineitems}
\phantomsection\label{\detokenize{source/pyzones:pyzones.Zone}}\pysiglinewithargsret{\sphinxbfcode{\sphinxupquote{class }}\sphinxcode{\sphinxupquote{pyzones.}}\sphinxbfcode{\sphinxupquote{Zone}}}{\emph{centre}, \emph{radius}, \emph{colour=None}}{}
Bases: {\hyperref[\detokenize{source/pyzones:pyzones.Circle}]{\sphinxcrossref{\sphinxcode{\sphinxupquote{pyzones.Circle}}}}}

A sound zone to be used in setup of the soundfield’s geometry
\begin{description}
\item[{centre}] \leavevmode{[}list{]}
A list of coordinates (x, y) describing the centre of the circle

\item[{radius}] \leavevmode{[}float{]}
The radius of the circle

\item[{colour}] \leavevmode{[}list{]}
A list of float values (r, g, b)

\end{description}
\index{colour (pyzones.Zone attribute)@\spxentry{colour}\spxextra{pyzones.Zone attribute}}

\begin{fulllineitems}
\phantomsection\label{\detokenize{source/pyzones:pyzones.Zone.colour}}\pysigline{\sphinxbfcode{\sphinxupquote{colour}}}
A list of float values (r, g, b)
\begin{quote}\begin{description}
\item[{Returns}] \leavevmode
A list of float values (r, g, b)

\item[{Return type}] \leavevmode
list

\end{description}\end{quote}

\end{fulllineitems}


\end{fulllineitems}

\index{convert\_to\_db() (in module pyzones)@\spxentry{convert\_to\_db()}\spxextra{in module pyzones}}

\begin{fulllineitems}
\phantomsection\label{\detokenize{source/pyzones:pyzones.convert_to_db}}\pysiglinewithargsret{\sphinxcode{\sphinxupquote{pyzones.}}\sphinxbfcode{\sphinxupquote{convert\_to\_db}}}{\emph{pressures}}{}
Convert a pressure or list of pressures to decibels
\begin{quote}\begin{description}
\item[{Parameters}] \leavevmode
\sphinxstyleliteralstrong{\sphinxupquote{pressures}} (\sphinxstyleliteralemphasis{\sphinxupquote{float}}) \textendash{} The input pressures

\item[{Returns}] \leavevmode
The resultant pressures in dB

\item[{Return type}] \leavevmode
float

\end{description}\end{quote}

\end{fulllineitems}



\chapter{Indices and tables}
\label{\detokenize{index:indices-and-tables}}\begin{itemize}
\item {} 
\DUrole{xref,std,std-ref}{genindex}

\item {} 
\DUrole{xref,std,std-ref}{modindex}

\item {} 
\DUrole{xref,std,std-ref}{search}

\end{itemize}


\renewcommand{\indexname}{Python Module Index}
\begin{sphinxtheindex}
\let\bigletter\sphinxstyleindexlettergroup
\bigletter{p}
\item\relax\sphinxstyleindexentry{pyzones}\sphinxstyleindexpageref{source/pyzones:\detokenize{module-pyzones}}
\end{sphinxtheindex}

\renewcommand{\indexname}{Index}
\printindex
\end{document}